\documentclass[a4paper]{article}

\usepackage[ngerman]{babel}
\usepackage[utf8]{inputenc}
\usepackage[left=2cm,right=2cm,top=2.5cm,bottom=2.5cm]{geometry}

\usepackage{float}
\usepackage{setspace}
\usepackage{mathabx}
\usepackage{scalerel}
\usepackage{graphicx}
\usepackage{csquotes}

\usepackage{tikz} 
\usetikzlibrary{graphs,graphs.standard}
\usepackage{tikzpagenodes}

\title{Mathematische Algorithmen\\ Planarität - Minoren und der Satz von Kuratowski}
\author{Sven Kube }
\date{10. Juni 2022}

\newtheorem{definition}{Definition}
\setlength{\parindent}{0pt}

\begin{document}

\maketitle
\begin{tikzpicture}[remember picture,overlay,shift={(current page.north east)}]
\node[anchor=north east,xshift=5pt,yshift=-2cm]{\includegraphics[width=1.7cm]{./FHAC.jpg}};
\end{tikzpicture}

\vspace{2cm}

\tableofcontents

\setstretch{1.25}

\newpage

\section{Motivation}
Bisher haben wir mit Eulers Polyedersatz eine Eigenschaft planarer Graphen kennengelernt.
Für den $K_{3,3}$ sowie den $K_5$ haben wir dabei einen Widerspruch für die Kantenabschätzung hergeleitet und gezeigt, dass der $K_{3,3}$ sowie der $K_5$ nicht planar sind.\\

Aus diesem Zusammenhang kann man leicht folgern, dass auch der $K_{3,3}$ oder der $K_5$ um Knoten und Kanten erweitert, nicht planar sein kann.
Man könnte nun auf den Gedanken kommen, den $K_{3,3}$ oder $K_5$ als \textit{atomare Bestandteile} von Graphen die nicht planar sind zu bezeichnen.
Als \textit{Bausteine für Nichtplanarität}.
Man könnte sagen: \enquote{Wenn ein Graph solch einen atomaren Bauteil \textit{drin hat}, ist er nicht planar}.\\

Die zuvor genannten Gedanken, die hier noch diffus formuliert sind, werden in den kommenden Kapiteln definiert, erläutert und anhand von Beispielen anschaulich gemacht. 

\section{Fast Planarität des $K_{3,3}$ und $K_5$}
In der Motivation haben wir bei dem $K_{3,3}$ und $K_5$ von \textit{atomaren Bausteinen für Nichtplanarität} gesprochen.
Wie bei Atomen in der Physik stellen wir uns dabei die kleinsten, nicht weiter aufteilbaren, Bausteine für Nichtplanarität vor.
Wir sprechen also, auf unsere Graphen bezogen, von Graphen, die nicht planar sind und ebenfalls die kleinsten nicht planaren Graphen sind.
Dazu betrachten wir nun die fast planaren Graphen.\\

Ein Graph $G$ ist fast planar, wenn er durch Löschen einer Kante planar wird.
Die folgenden Abbildungen zeigen, dass dies für den $K_5$ und den $K_{3,3}$ gilt.

\begin{figure}[H]
    \centering
    \begin{minipage}{.4\linewidth}
        \centering
   \vspace{-1cm}
 \begin{tikzpicture}[node distance={30mm}, thick, main/.style = {draw, circle}, highlight/.style = {draw=blue, circle,text=blue}]

    \node[main] at (0, 2.3) (1) {1};
    \node[main] at (2.3, 0.4) (2) {2};
    \node[main] at (1.5, -2) (3) {3};
    \node[main] at (-1.5, -2) (4) {4};
    \node[main] at (-2.3, 0.4) (5) {5};


    \path (1) edge (2);
    \path[blue] (1) edge (3);
    \path (1) edge (4);
    \path (2) edge (3);
    \path (2) edge (4);
    \path (2) edge (5);
    \path (3) edge (4);
    \path (3) edge (5);
    \path (4) edge (5);
    \path (5) edge (1);
    
\end{tikzpicture} 
\end{minipage}
\begin{minipage}{.07\linewidth}
\centering
\vspace{-1cm}
$\Longrightarrow$
\end{minipage}
    \begin{minipage}{.4\linewidth}
        \centering
      
    \begin{tikzpicture}[node distance={30mm}, thick, main/.style = {draw, circle}, highlight/.style = {draw=blue, circle,text=blue}]

    \node[main] at (0, 2.3) (1) {1};
    \node[main] at (2.3, 0.4) (2) {2};
    \node[main] at (1.5, -2) (3) {3};
    \node[main] at (-1.5, -2) (4) {4};
    \node[main] at (-2.3, 0.4) (5) {5};


    \path (1) edge (2);
    \path (1) edge [out=160,in=180, distance=3cm ] (4);
    \path (2) edge (3);
    \path (2) edge [out=300,in=320, distance=3cm ] (4);
    \path (2) edge (5);
    \path (3) edge (4);
    \path (3) edge (5);
    \path (4) edge (5);
    \path (5) edge (1);
    

\end{tikzpicture} 
\end{minipage}
\vspace{-1cm}
    \caption{Durch Entfernen einer Kante wird der $K_5$ planar.}
    \label{fig:my_label}
\end{figure}

\vspace{-1.8cm}

\begin{figure}[H]
    \centering
    \begin{minipage}{.3\linewidth}
    \centering
    \vspace{-2.7cm}
        \begin{tikzpicture}[node distance={30mm}, thick, main/.style = {draw, circle}, highlight/.style = {draw=blue, circle,text=blue}]

    \node[main] at (0, 0) (1) {1};
    \node[main] at (2, 0) (2) {2};
    \node[main] at (4, 0) (3) {3};
    
    \node[main] at (0, -2) (4) {4};
    \node[main] at (2, -2) (5) {5};
    \node[main] at (4, -2) (6) {6};

    \path (1) edge (4);
    \path (1) edge (5);
    \path (1) edge (6);
    
    \path (2) edge (4);
    \path (2) edge (5);
    \path (2) edge (6);
    
    \path[highlight] (3) edge (4);
    \path (3) edge (5);
    \path (3) edge (6);
    
\end{tikzpicture} 
    \end{minipage}
    \begin{minipage}{.07\linewidth}
    \centering
    \vspace{-2.7cm}
    $\Longrightarrow$
    \end{minipage}
    \hspace{-1.5cm}
    \begin{minipage}{.5\linewidth}
    \centering
    \begin{tikzpicture}[node distance={30mm}, thick, main/.style = {draw, circle}, highlight/.style = {draw=blue, circle,text=blue}]

    \node[main] at (0, 0) (1) {1};
    \node[main] at (2, 0) (2) {2};
    \node[main] at (4, 0) (3) {3};
    
    \node[main] at (0, -2) (4) {4};
    \node[main] at (2, -2) (5) {5};
    \node[main] at (4, -2) (6) {6};

    \path (1) edge (4);
    \path (1) edge (5);
    \path (1) edge (6);
    
    \path (2) edge [out=130,in=180, distance=3cm ] (4);
    \path (2) edge [out=30,in=300, distance=7cm ] (5);
    \path (2) edge (6);
    
    \path (3) edge [out=330,in=310, distance=3cm ] (5);
    \path (3) edge (6);
    
    \end{tikzpicture} 
    \end{minipage}
    \vspace{-4cm}
    \caption{Durch Entfernen einer Kante wird der $K_{3,3}$ planar.}
    \label{fig:my_label}
\end{figure}

Daraus folgt, dass der $K_{3,3}$ und der $K_5$ fast planar sind.


\section{Begriffe aus der Graphentheorie}
\label{section:begriffe_der_graphentheorie}
Für die in kommenden Kapiteln folgenden Definitionen und Erklärungen definieren wir nun zunächst den Begriff der \textit{Kantenkontraktion} und des \textit{Unterteilungsgraphs} aus der Graphentheorie.

\subsection{Kantenkontraktion}
Als Kontraktion oder auch Kantenkontraktion wird in der Graphentheorie eine grundlegende Operation auf einem Graph beschrieben.

\begin{definition}[Kantenkontraktion]
Aus einem Graph $G$ wird eine Kante $e$ entfernt und die beiden vorher durch die Kante $e$ verbundenen Knoten $u$ und $v$ zu einem neuen Knoten $w$ vereinigt.
\end{definition}

\begin{figure}[H]
    \centering
    \begin{tabular}{@{}c@{}}
    \begin{tikzpicture}[node distance={30mm}, thick, main/.style = {draw, circle}, highlight/.style = {draw=blue, circle,text=blue}]
    
        \node[main] at (-2, 1)  (1)    {};
        \node[main] at (1.5, 0.8)  (2)    {};
        
        \node[highlight] at (0, 0)  (u)    {u};
        \node[highlight] at (0, -2)  (v)    {v};
        
        \node[main] at (-1, -3.5)  (3)    {};
        \node[main] at (0.3, -3.5)  (4)    {};
        \node[main] at (1.5, -2)  (5)    {};
    
        
        \path (1) edge (2);
        \path (1) edge (u);
        \path (2) edge (u);
        
        \path[blue] (u) edge node[right] {$e$} (v);
        
        \path (3) edge (v);
        \path (4) edge (v);
        \path (5) edge (v);
        
        \path (4) edge (5);
        
        \path[dashed] (1) edge  (-2.5, 1.5);
        \path[dashed] (1) edge  (-0.8, 1.6);
        
        \path[dashed] (2) edge  (1.5, 1.3);
        \path[dashed] (2) edge  (2, 0.8);
        
        
        \path[dashed] (3) edge  (-1.5, -3.5);
        \path[dashed] (3) edge  (-1, -4);
        \path[dashed] (4) edge  (0.3, -4);
       
        
        \path[dashed] (5) edge  (2, -2.5);
        \path[dashed] (5) edge  (2, -1.5);
        
        \draw [-stealth,draw=green](-0.2,-0.4) -- (-0.2,-0.9);
        \draw [-stealth,draw=green](-0.2,-1.6) -- (-0.2,-1.1);

    
    \end{tikzpicture} 
    \end{tabular}
    \hspace{0.5cm}
    \begin{tabular}{p{1cm}}
    \begin{center}
         $\Longrightarrow$
    \end{center}
    \end{tabular}
    \hspace{0.5cm}
    \begin{tabular}{@{}c@{}}
   \begin{tikzpicture}[node distance={30mm}, thick, main/.style = {draw, circle}, highlight/.style = {draw=blue, circle,text=blue}]
    
        \node[main] at (-2, 1)  (1)    {};
        \node[main] at (1.5, 0.8)  (2)    {};
        
        \node[highlight] at (0, -1)  (w)    {w};
        
        \node[main] at (-1, -3.5)  (3)    {};
        \node[main] at (0.3, -3.5)  (4)    {};
        \node[main] at (1.5, -2)  (5)    {};
    
        
        \path (1) edge (2);
        \path (1) edge (w);
        \path (2) edge (w);
        
        
        \path (3) edge (w);
        \path (4) edge (w);
        \path (5) edge (w);
        
        \path (4) edge (5);
        
        \path[dashed] (1) edge  (-2.5, 1.5);
        \path[dashed] (1) edge  (-0.8, 1.6);
        
        \path[dashed] (2) edge  (1.5, 1.3);
        \path[dashed] (2) edge  (2, 0.8);
        
        
        \path[dashed] (3) edge  (-1.5, -3.5);
        \path[dashed] (3) edge  (-1, -4);
        \path[dashed] (4) edge  (0.3, -4);
       
        
        \path[dashed] (5) edge  (2, -2.5);
        \path[dashed] (5) edge  (2, -1.5);
        
      

    
    \end{tikzpicture}
    \end{tabular}
    \caption{Durch Kantenkontraktion von $\{u,v\}$ entsteht ein neuer Knoten $w$.}
    \label{fig:edge_contraction}
\end{figure}

In der Abbildung \ref{fig:edge_contraction} ist beispielhaft die Kontraktion der Kante $e$ dargestellt.
Die beiden inzidenten Knoten $u$ und $v$ der Kante $e$ werden nach der Definition der Kantenkontraktion zu einem neuen Knoten $w$ vereinigt.
Die mit $u$ und $v$ verbundenen Kanten werden zu dem neuen Knoten $w$ umgelenkt.

\subsection{Unterteilungsgraph}
Als Unterteilungsgraph wird in der Graphentheorie ein Graph bezeichnet, der durch Kantenunterteilung aus einem anderen Graphen entstanden ist.\\

Sei $G=(V,E)$ ein ungerichteter Graph, dann versteht man unter einer Unterteilung einer Kante $e=\{u, v\} \in E$ die Ersetzung dieser Kante durch zwei neue Kanten $e_1$ und $e_2$, die die beiden Knoten $u$ und $v$ der entfernten Kante mit einem neuen Knoten $w \notin V$ verbinden. Auf diese Weise entsteht ein neuer Graph $G' = (V', E')$ mit der neuen Knotenmenge $V'=V \cup \{ w \}$ und der neuen Kantenmenge: $E' = E \setminus \{ e \} \cup \{ e_1, e_2 \}$, wobei $e_1=\{ u, w \}$ und $e_2=\{ w, v \}$ sind.\\

\begin{definition}[Unterteilungsgraph]
Als Unterteilungsgraph eines Graphen $G$ wird ein Graph bezeichnet, der durch (null-, ein- oder mehrmalige) Kantenunterteilung in $G$ entsteht.
\end{definition}

\begin{figure}[H]
    \centering
    $G:$\hspace{0.5cm}
        \begin{tabular}{@{}c@{}}

    \begin{tikzpicture}[node distance={30mm}, thick, main/.style = {draw, circle}]
        \node[main] (1) {$u$}; 
        \node[main] (2) [below of=1] {$v$}; 

        \path (1) edge  node [right] {$e$} (2);
    \end{tikzpicture}
    \end{tabular}
    \hspace{1cm}$\Longrightarrow$\hspace{1cm}
    $G':$\hspace{0.5cm}
        \begin{tabular}{@{}c@{}}
    \begin{tikzpicture}[node distance={15mm}, thick, main/.style = {draw, circle}]
        \node[main] (1) {$u$}; 
        \node[draw, circle, blue] (2) [below of=1] {$w$}; 
        \node[main] (3) [below of=2] {$v$}; 

       \path[blue] (1) edge node [right] {$e_1$} (2);
       \path[blue] (2) edge node [right] {$e_2$} (3);
    
    \end{tikzpicture} 
    \end{tabular}
    \caption{$G$ wird durch Kantenunterteilung von $e$ zu $G'$.}
    \label{fig:edge_subdivision}
\end{figure}

In der Abbildung \ref{fig:edge_subdivision} wird beispielhaft eine Kantenunterteilung der Kante $e$ gezeigt.
Durch die Unterteilung von $e$ entsteht ein neuer Knoten $w$.
Aus der vorherigen Kante $\{u,v\}$ sind nach der Unterteilung in dem Graph $G'$ die Kanten $E = \{ \{u,w\}\{w,u\} \}$ ($e_1$ und $e_2$) entstanden.


\section{Minoren}
Mithilfe der in Abschnitt \ref{section:begriffe_der_graphentheorie} vorgestellten Begriffen der Graphentheorie, der Kantenkontraktion und dem Unterteilungsgraph, wollen wir nun die Begriffe des \textit{Minors} sowie des \textit{topologischen Minors} einführen und definieren.

\subsection{Minor}
Im Folgenden wollen wir uns der Definition des Minors anhand eines Beispiels nähern.\\

Betrachten wir folgendes Beispiel bestehend aus den drei Graphen $K_3$, $G_1$ und $G_2$.

\begin{figure}[H]
    \centering
    \begin{minipage}{.32\linewidth}
    \centering
\begin{tikzpicture}[node distance={30mm}, thick, main/.style = {draw, circle, fill}, highlight/.style = {draw=blue, circle,text=blue}]

    \node[main] at (0, 0) (1) {};
    \node[main] at (-1, -2) (2) {};
    \node[main] at (1, -2) (3) {};

    \path (1) edge (2);
    \path (2) edge (3);
    \path (3) edge (1);

\end{tikzpicture}
\vspace{0.6cm}
  \caption{$K_3$}

\end{minipage}
    \begin{minipage}{.32\linewidth}
    \centering
 \begin{tikzpicture}[node distance={30mm}, thick, main/.style = {draw, circle, fill}, small/.style = {draw, circle, fill, scale = 0.4}]

    \node[small] at (0, 0) (11) {};
    \node[small] at (0.3, -0.3) (12) {};
    \node[small] at (0.4, 0.2) (13) {};

    
    
    \node[small] at (-1, -2) (21) {};
    \node[small] at (-0.6, -1.9) (22) {};
    
    
    
    \node[small] at (1, -2) (31) {};
    \node[small] at (1.15, -1.3) (32) {};
    \node[small] at (1.5, -1.3) (33) {};
    \node[small] at (1.5, -1.8) (34) {};
    

    \path (11) edge (12);
    \path (11) edge (13);
    \path (11) edge (21);
    \path (32) edge (12);
    
    \path (32) edge (33);
    \path (32) edge (31);
    \path (34) edge (33);
    \path (34) edge (31);
    \path (31) edge (33);
    
    \path (22) edge (31);
    \path (22) edge (21);
    
    \draw (0.2,0) circle (15pt);
    
    \draw (-0.8, -1.95) ellipse (15pt and 10pt);
    
    \draw (1.25, -1.7) circle (20pt);

\end{tikzpicture}
  \caption{$G_1$}

\end{minipage}
\begin{minipage}{.32\linewidth}
\centering
 \begin{tikzpicture}[node distance={30mm}, thick, main/.style = {draw, circle, fill}, small/.style = {draw, circle, fill, scale = 0.4},gray/.style = {draw=gray, circle, fill=gray, scale = 0.4}]

    \node[small] at (0, 0) (11) {};
    \node[small] at (0.3, -0.3) (12) {};
    \node[small] at (0.4, 0.2) (13) {};
    
    \node[gray] at (1.2, -0.3) (X1) {};
    \path[gray] (13) edge (X1);
    \path[gray] (33) edge (X1);

    
    
    \node[small] at (-1, -2) (21) {};
    \node[small] at (-0.6, -1.9) (22) {};
    
    \node[gray] at (-0.8, -0.4) (X2) {};
    \node[gray] at (-1.7, -2) (X3) {};
    \node[gray] at (-1.3, -1.5) (X4) {};
    
    \path[gray] (11) edge (X2);
    \path[gray] (X2) edge (X4);
    \path[gray] (X3) edge (X4);
    
    \path[gray] (X3) edge (21);
    \path[gray] (X4) edge (21);

    
    
    \node[small] at (1, -2) (31) {};
    \node[small] at (1.15, -1.3) (32) {};
    \node[small] at (1.5, -1.3) (33) {};
    \node[small] at (1.5, -1.8) (34) {};
    

    \path (11) edge (12);
    \path (11) edge (13);
    \path (11) edge (21);
    \path (32) edge (12);
    
    \path (32) edge (33);
    \path (32) edge (31);
    \path (34) edge (33);
    \path (34) edge (31);
    \path (31) edge (33);
    
    \path (22) edge (31);
    \path (22) edge (21);
    
    \draw (0.2,0) circle (15pt);
    
    \draw (-0.8, -1.95) ellipse (15pt and 10pt);
    
    \draw (1.25, -1.7) circle (20pt);

\end{tikzpicture} 
  \caption{$G_2$}

\end{minipage}
\end{figure}

Auf der linken Seite ist der vollständige Graph mit drei Knoten $K_3$ abgebildet. 
Diesen erhält man durch die Kontraktion von Kanten innerhalb des Graphen $G_1$. 
Der Graph $G_1$ ist selbst wieder in dem Graph $G_2$ enthalten.
Dies ist ein Beispiel dafür, dass der $K_3$ ein Minor von $G_2$ ist.

\begin{definition}[Minor]
Ein Graph $G$ heißt \textit{Minor} von $G_2$, wenn $G_2$ einen Teilgraph enthält, aus dem durch Kantenkontraktion $G$ hervorgeht.
\end{definition}


\subsection{Topologischer Minor}
Zusätzlich zu dem Begriff des Minors wollen wir nun auch den Begriff des \textit{topologischen Minors} einführen.\\

Betrachten wir folgendes Beispiel bestehend aus den drei Graphen $K_3$, $G_1$ und $G_2$.

\begin{figure}[H]
    \centering
    \begin{minipage}{.32\linewidth}
    \centering
\begin{tikzpicture}[node distance={30mm}, thick, main/.style = {draw, circle, fill}, highlight/.style = {draw=blue, circle,text=blue}]

    \node[main] at (0, 0) (1) {};
    \node[main] at (-1, -2) (2) {};
    \node[main] at (1, -2) (3) {};

    \path (1) edge (2);
    \path (2) edge (3);
    \path (3) edge (1);

\end{tikzpicture}
  \caption{$K_3$}

\end{minipage}
    \begin{minipage}{.32\linewidth}
    \centering
 \begin{tikzpicture}[node distance={30mm}, thick, main/.style = {draw, circle, fill}, small/.style = {draw, circle, fill, scale = 0.4}]

    \node[main] at (0, 0) (1) {};
    
    \node[small] at (-0.5, -1) (a) {};
    
    \node[main] at (-1, -2) (2) {};
    
    \node[small] at (0.65, -1.3) (a) {};
    
    \node[main] at (1, -2) (3) {};
    
    \node[small] at (-0.5, -2) (c) {};
    \node[small] at (0.3, -2) (d) {};

    \path (1) edge (2);
    \path (2) edge (3);
    \path (3) edge (1);

\end{tikzpicture}
  \caption{$G_1$}

\end{minipage}
\begin{minipage}{.32\linewidth}
\centering
\begin{tikzpicture}[node distance={30mm}, thick, main/.style = {draw, circle, fill}, small/.style = {draw, circle, fill, scale = 0.4}, gray/.style = {draw=gray, circle, fill=gray,}]

    \node[main] at (0, 0) (1) {};
    
    
    \node[small] at (-0.5, -1) (a) {};
    
    \node[main] at (-1, -2) (2) {};
    
    \node[small] at (0.65, -1.3) (a) {};
    
    \node[main] at (1, -2) (3) {};
    
    \node[small] at (-0.5, -2) (c) {};
    \node[small] at (0.3, -2) (d) {};
    
    
    
    \node[gray] at (-1.5, 0) (z) {};
    \node[gray] at (2, 0) (y) {};
    \node[gray] at (2.5, -2) (x) {};

    

    \path (1) edge (2);
    \path (2) edge (3);
    \path (3) edge (1);
    
    \path[gray] (3) edge (x);
    \path[gray] (1) edge (y);
    \path[gray] (3) edge (y);
    \path[gray] (1) edge (z);

\end{tikzpicture} 
  \caption{$G_2$}

\end{minipage}
\end{figure}


Der Graph $G_1$ in der Mitte stellt einen Unterteilungsgraph von dem Graph $K_3$ auf der linken Seite dar. Rechts dargestellt ist der Graph $G_2$ in dem der Unterteilungsgraph $G_1$ enthalten ist. $K_3$ ist topologischer Minor von $G_2$.

\begin{definition}[Topologischer Minor]
Ein Graph $G$ heißt \textit{topologischer Minor} von $G_2$, wenn $G_2$ einen Unterteilungsgraphen von $G$ enthält.
\end{definition}

\newpage

\subsection{Minorenrelation}
Durch die Einführung des Begriffs \textit{Minor} können wir die Minorenrelation definieren:

\begin{definition}
$ G \preccurlyeq H :\Longleftrightarrow G$ ist Minor von $H$.
\end{definition}

Eigenschaften der Minorenrelation:
\begin{itemize}
    \item Die Minorenrelation ist reflexiv, transitiv und antisymetrisch und definiert daher eine Ordnungsrelation auf den endlichen Graphen.
    \item Es gilt, dass jeder topologische Minor eines Graphen $G$ auch ein Minor von $G$ ist.
    \item Es gilt \textbf{nicht}, dass jeder Minor eines Graphen $G$ auch ein topologischer Minor von $G$ ist.
\end{itemize}

\vspace{2em}

Im Folgenden betrachten wir ein Beispiel anhand dessen klar wird, dass nicht jeder Minor eines Graphen $G$ auch ein topologischer Minor von $G$ ist.

\begin{figure}[H]
    \centering
    \begin{minipage}{.32\linewidth}
    \centering

\begin{tikzpicture}[node distance={30mm}, thick, main/.style = {draw, circle, fill}, highlight/.style = {draw=blue, circle,text=blue}]

    \node[main] at (0, 0) (1) {};
    \node[main] at (-3, 0) (2) {};
    \node[main] at (0, -3) (3) {};
    \node[main] at (-3, -3) (4) {};
    
    \node[main] at (-1.5, -1.5) (z) {};
    
    \node[draw=blue, circle,fill=blue] at (-0.75, -0.75) (a) {};
    \node[draw=blue, circle,fill=blue] at (-2.25, -0.75) (b) {};
    \node[draw=blue, circle,fill=blue] at (-0.75, -2.25) (c) {};
    \node[draw=blue, circle,fill=blue] at (-2.25, -2.25) (d) {};

    \path (1) edge (a);
    \path (2) edge (b);
    \path (3) edge (c);
    \path (4) edge (d);
    
    \path (z) edge (a);
    \path (z) edge (b);
    \path (z) edge (c);
    \path (z) edge (d);

    
    \path (1) edge (2);
    \path (1) edge (3);
    
    \path (4) edge (2);
    \path (4) edge (3);

\end{tikzpicture}
  \caption{$F$}
  
\end{minipage}
    \begin{minipage}{.32\linewidth}
    \centering

\begin{tikzpicture}[node distance={30mm}, thick, main/.style = {draw, circle, fill}, highlight/.style = {draw=blue, circle,text=blue}]

    \node[main] at (0, 0) (1) {};
    \node[main] at (-3, 0) (2) {};
    \node[main] at (0, -3) (3) {};
    \node[main] at (-3, -3) (4) {};
    
    \node[draw=blue, circle,fill=blue] at (-1.5, -1.5) (5) {};

    \path (1) edge (5);
    \path (2) edge (5);
    \path (3) edge (5);
    \path (4) edge (5);
    
    \path (1) edge (2);
    \path (1) edge (3);
    
    \path (4) edge (2);
    \path (4) edge (3);

\end{tikzpicture}
  \caption{$G$}

\end{minipage}
    \begin{minipage}{.32\linewidth}
    \centering

 \begin{tikzpicture}[node distance={30mm}, thick, main/.style = {draw, circle, fill}, highlight/.style = {draw=blue, circle,text=blue}]

    \node[main] at (0, 0) (1) {};
    \node[main] at (-3, 0) (2) {};
    \node[main] at (0, -3) (3) {};
    \node[main] at (-3, -3) (4) {};
    
    \node[draw=blue, circle,fill=blue] at (-1, -1) (a) {};
    \node[draw=blue, circle,fill=blue] at (-2, -1) (b) {};
    \node[draw=blue, circle,fill=blue] at (-1, -2) (c) {};
    \node[draw=blue, circle,fill=blue] at (-2, -2) (d) {};

    \path (a) edge (b);
    \path (a) edge (c);
    \path (d) edge (b);
    \path (d) edge (c);
    
    \path (1) edge (a);
    \path (2) edge (b);
    \path (3) edge (c);
    \path (4) edge (d);
    
    \path (1) edge (2);
    \path (1) edge (3);
    
    \path (4) edge (2);
    \path (4) edge (3);

\end{tikzpicture}
  \caption{$H$}
  
\end{minipage}
\end{figure}

Zu Beginn betrachten wir die Graphen $G$ und $F$ und versuchen zu zeigen, dass $G$ sowohl ein Minor als auch ein topologischer Minor von $F$ ist:
\begin{itemize}
    \item Wenn wir in dem Graph $F$ die Kante zwischen den blau markierten Knoten und den Eckpunkten kontrahieren, erhalten wir den Graph $G$.\\
    $\Rightarrow$ $G$ ist Minor von $F$.
    \item Der Graph $F$ ist der Unterteilungsgraph von $G$, bei dem jede Kante vom Mittelpunkt zu den Eckpunkten unterteilt wurde.\\
    $\Rightarrow$ $G$ ist topologischer Minor von $F$.
\end{itemize}

Betrachten wir nun die Graphen $G$ und $H$ und versuchen zu zeigen, dass $G$ sowohl ein Minor als auch ein topologischer Minor von $H$ ist
\begin{itemize}
    \item Wenn wir in dem Graphen $H$ die Kanten zwischen den blauen Knoten kontrahieren, sodass nur ein blauer Knoten verbleibt, haben wir den Graph $G$ erzeugt.\\
    $\Rightarrow$ $G$ ist Minor von $H$.
    \item Bei dem Versuch in dem Graph $H$ einen Unterteilungsgraphen von $G$ zu finden haben wir nun ein Problem. Der Graph $H$ enthält keinen Unterteilungsgraph von $G$. $G$ ist daher kein topologischer Minor von $H$.\\
    $\Rightarrow$ $G$ ist \textbf{kein} topologischer Minor von $H$.
\end{itemize}

Damit haben wir ein Beispiel gefunden, bei dem ein Graph ein Minor eines anderen Graphen ist, es sich aber nicht auch um einen topologischen Minor handelt.

\newpage

\section{Kuratowski}
Bisher haben wir mit Eulers Polyedersatz eine Eigenschaft planarer Graphen kennengelernt.
Im Folgenden werden wir anhand der im vorherigen Abschnitt definierten Begriffe \textit{Minor} und \textit{topologischer Minor} eine Methode kennenlernen, um nachzuweisen, ob ein Graph planar ist oder nicht.\\

\subsection{Satz von Kuratowski}
Mit Hilfe der zuvor definierten Minoren und den Graphen $K_{3,3}$ und $K_5$ lassen sich nun alle planaren Graphen charakterisieren.\\

1930 hat dazu der polnische Mathematiker Kazimierz Kuratowski den folgenden Satz verfasst:\\

\textbf{Satz von Kuratowski:}
\textit{Ein Graph ist genau dann planar, wenn er weder einen $K_{3,3}$ noch einen $K_5$ als Minor enthält.}\\


Alternativ kann man formulieren, dass ein Graph genau dann planar ist, wenn er weder den $K_{3,3}$ noch den $K_5$ als \textit{topologischen Minor} enthält.\\

Durch den Satz von Kuratowski lassen sich zwei Aussagen treffen:

\begin{enumerate}
    \item Wenn ein Graph $G$ den $K_{3,3}$ oder $K_5$ als Minor enthält, ist der Graph $G$ \textbf{nicht} planar.
    \item Wenn ein Graph $G$ \textbf{nicht} planar ist, enthält er den $K_{3,3}$ oder $K_5$ als Minor.
\end{enumerate}

Dass der erstgenannte Zusammenhang gilt kann man sich leicht überlegen.
Wenn man zu einem Graphen $G$, der bereits den $K_{3,3}$ oder $K_5$ enthält und daher nicht planar ist, weitere Knoten und Kanten hinzufügt, kann der Graph $G$ dadurch nicht planar werden.\\

Im Vergleich dazu ist nicht intuitiv klar, dass auch der zweite Zusammenhang, wenn ein Graph $G$ \textbf{nicht} planar ist er den $K_{3,3}$ oder $K_5$ als Minor enthält, allgemein gilt.\\

\textit{Bemerkung:} Der Beweis des Satzes von Kuratowski wird hier aufgrund seiner Länge nicht geführt.


\subsection{Anwendung beim Petersen-Graph}
In den folgenden Abbildungen sind die Knotenlöschungen und Kantenkontraktionen dargestellt, mit denen man aus dem Petersen-Graph den $K_{3,3}$ erhält.
Damit ist, wie bereits zuvor erwähnt, mithilfe des Satzes von Kuratowski nachgewiesen, dass der Petersen-Graph nicht planar ist.

\begin{figure}[H]
    \centering
    \begin{tikzpicture}[node distance={30mm}, thick, main/.style = {draw, circle}, highlight/.style = {draw=blue, circle,text=blue}]

    \node[main] at (0, 2.3) (1) {1};
    \node[main] at (2.3, 0.4) (2) {2};
    \node[main] at (1.5, -2) (3) {3};
    \node[main] at (-1.5, -2) (4) {4};
    \node[main] at (-2.3, 0.4) (5) {5};

    \node[main] at (0, 1) (a) {a};
    \node[main] at (1, 0.2) (b) {b};
    \node[main] at (0.7, -1) (c) {c};
    \node[main] at (-0.7, -1) (d) {d};
    \node[main] at (-1, 0.2) (e) {e};

    \path (1) edge (2);
    \path (2) edge (3);
    \path (3) edge (4);
    \path (4) edge (5);
    \path (5) edge (1);
    
    \path (a) edge (c);
    \path (a) edge (d);
    \path (b) edge (e);
    \path (b) edge (d);
    \path (c) edge (e);
    
    \path (1) edge (a);
    \path (2) edge (b);
    \path (3) edge (c);
    \path (4) edge (d);
    \path (5) edge (e);

\end{tikzpicture} 
    \caption{Petersen-Graph}
    \label{fig:my_label}
\end{figure}

Folgend wird in 5 Schritten mithilfe von Knotenlöschungen und Kantenkontraktionen gezeigt, dass der $K_{3,3}$ Minor des Petersen-Graph ist.

\begin{figure}[H]
    \centering
    \begin{minipage}{.39\linewidth}
    \centering
\begin{tikzpicture}[node distance={30mm}, thick, main/.style = {draw, circle}, highlight/.style = {draw=blue, circle,text=blue}]

    \node[main] at (0, 2.3) (1) {1};
    \node[main] at (2.3, 0.4) (2) {2};
    \node[main] at (1.5, -2) (3) {3};
    \node[main] at (-1.5, -2) (4) {4};
    \node[main] at (-2.3, 0.4) (5) {5};

    \node[main] at (0, 1) (a) {a};
    \node[highlight] at (1, 0.2) (b) {b};
    \node[main] at (0.7, -1) (c) {c};
    \node[main] at (-0.7, -1) (d) {d};
    \node[main] at (-1, 0.2) (e) {e};

    \path (1) edge (2);
    \path (2) edge (3);
    \path (3) edge (4);
    \path (4) edge (5);
    \path (5) edge (1);
    
    \path (a) edge (c);
    \path (a) edge (d);
    \path[highlight] (b) edge (e);
    \path[highlight] (b) edge (d);
    \path (c) edge (e);
    
    \path (1) edge (a);
    \path[highlight] (2) edge (b);
    \path (3) edge (c);
    \path (4) edge (d);
    \path (5) edge (e);

\end{tikzpicture} 
\end{minipage}
\begin{minipage}{.07\linewidth}
\centering
$\Longrightarrow$
\end{minipage}
\begin{minipage}{.39\linewidth}
\centering
\begin{tikzpicture}[node distance={30mm}, thick, main/.style = {draw, circle}, highlight/.style = {draw=blue, circle,text=blue}]

    \node[main] at (0, 2.3) (1) {1};
    \node[main] at (2.3, 0.4) (2) {2};
    \node[main] at (1.5, -2) (3) {3};
    \node[main] at (-1.5, -2) (4) {4};
    \node[main] at (-2.3, 0.4) (5) {5};

    \node[main] at (0, 1) (a) {a};
    \node[main] at (0.7, -1) (c) {c};
    \node[main] at (-0.7, -1) (d) {d};
    \node[main] at (-1, 0.2) (e) {e};

    \path (1) edge (2);
    \path (2) edge (3);
    \path (3) edge (4);
    \path (4) edge (5);
    \path (5) edge (1);
    
    \path (a) edge (c);
    \path (a) edge (d);
    \path (c) edge (e);
    
    \path (1) edge (a);
    \path (3) edge (c);
    \path (4) edge (d);
    \path (5) edge (e);

\end{tikzpicture} 
\end{minipage}
    \caption{Schritt 1: Knoten b wird gelöscht.}
    \label{fig:my_label}
\end{figure}

\begin{figure}[H]
    \centering
    \begin{minipage}{.39\linewidth}
    \centering
    \begin{tikzpicture}[node distance={30mm}, thick, main/.style = {draw, circle}, highlight/.style = {draw=blue, circle,text=blue}]

    \node[main] at (0, 2.3) (1) {1};
    \node[main] at (2.3, 0.4) (2) {2};
    \node[main] at (1.5, -2) (3) {3};
    \node[main] at (-1.5, -2) (4) {4};
    \node[main] at (-2.3, 0.4) (5) {5};

    \node[main] at (0, 1) (a) {a};
    \node[main] at (0.7, -1) (c) {c};
    \node[highlight] at (-0.7, -1) (d) {d};
    \node[main] at (-1, 0.2) (e) {e};

    \path (1) edge (2);
    \path (2) edge (3);
    \path (3) edge (4);
    \path (4) edge (5);
    \path (5) edge (1);
    
    \path (a) edge (c);
    \path[highlight] (a) edge (d);
    \path (c) edge (e);
    
    \path (1) edge (a);
    \path (3) edge (c);
    \path (4) edge (d);
    \path (5) edge (e);

\end{tikzpicture} 
    \end{minipage}
    \begin{minipage}{.07\linewidth}
    \centering
    $\Longrightarrow$
    \end{minipage}
    \begin{minipage}{.39\linewidth}
    \centering
    \begin{tikzpicture}[node distance={30mm}, thick, main/.style = {draw, circle}, highlight/.style = {draw=blue, circle,text=blue}]

    \node[main] at (0, 2.3) (1) {1};
    \node[main] at (2.3, 0.4) (2) {2};
    \node[main] at (1.5, -2) (3) {3};
    \node[main] at (-1.5, -2) (4) {4};
    \node[main] at (-2.3, 0.4) (5) {5};

    \node[main] at (0, 1) (a) {a};
    \node[main] at (0.7, -1) (c) {c};
    \node[main] at (-1, 0.2) (e) {e};

    \path (1) edge (2);
    \path (2) edge (3);
    \path (3) edge (4);
    \path (4) edge (5);
    \path (5) edge (1);
    
    \path (a) edge (c);    \path (c) edge (e);
    
    \path (1) edge (a);
    \path (3) edge (c);
    \path (4) edge (a);
    \path (5) edge (e);

\end{tikzpicture} 
    \end{minipage}
    \caption{Schritt 2: Kante zwischen Knoten d und Knoten a kontrahieren.}
    \label{fig:my_label}
\end{figure}

\begin{figure}[H]
    \centering
    \begin{minipage}{.39\linewidth}
    \centering
    \begin{tikzpicture}[node distance={30mm}, thick, main/.style = {draw, circle}, highlight/.style = {draw=blue, circle,text=blue}]

    \node[main] at (0, 2.3) (1) {1};
    \node[highlight] at (2.3, 0.4) (2) {2};
    \node[main] at (1.5, -2) (3) {3};
    \node[main] at (-1.5, -2) (4) {4};
    \node[main] at (-2.3, 0.4) (5) {5};

    \node[main] at (0, 1) (a) {a};
    \node[main] at (0.7, -1) (c) {c};
    \node[main] at (-1, 0.2) (e) {e};

    \path (1) edge (2);
    \path[highlight] (2) edge (3);
    \path (3) edge (4);
    \path (4) edge (5);
    \path (5) edge (1);
    
    \path (a) edge (c);    \path (c) edge (e);
    
    \path (1) edge (a);
    \path (3) edge (c);
    \path (4) edge (a);
    \path (5) edge (e);

\end{tikzpicture}
    \end{minipage}
    \begin{minipage}{.07\linewidth}
    \centering
    $\Longrightarrow$
    \end{minipage}
    \begin{minipage}{.39\linewidth}
    \centering
    \begin{tikzpicture}[node distance={30mm}, thick, main/.style = {draw, circle}, highlight/.style = {draw=blue, circle,text=blue}]

    \node[main] at (0, 2.3) (1) {1};
    \node[main] at (1.5, -2) (3) {3};
    \node[main] at (-1.5, -2) (4) {4};
    \node[main] at (-2.3, 0.4) (5) {5};

    \node[main] at (0, 1) (a) {a};
    \node[main] at (0.7, -1) (c) {c};
    \node[main] at (-1, 0.2) (e) {e};

    \path (1) edge (3);
    \path (3) edge (4);
    \path (4) edge (5);
    \path (5) edge (1);
    
    \path (a) edge (c);    
    \path (c) edge (e);
    
    \path (1) edge (a);
    \path (3) edge (c);
    \path (4) edge (a);
    \path (5) edge (e);

\end{tikzpicture} 
    \end{minipage}

    \caption{Schritt 3: Kante zwischen Knoten 2 und Knoten 3 kontrahieren.}
    \label{fig:my_label}
\end{figure}


\begin{figure}[H]
    \centering
    \begin{minipage}{.39\linewidth}
    \centering
    \begin{tikzpicture}[node distance={30mm}, thick, main/.style = {draw, circle}, highlight/.style = {draw=blue, circle,text=blue}]

    \node[main] at (0, 2.3) (1) {1};
    \node[main] at (1.5, -2) (3) {3};
    \node[main] at (-1.5, -2) (4) {4};
    \node[main] at (-2.3, 0.4) (5) {5};

    \node[main] at (0, 1) (a) {a};
    \node[main] at (0.7, -1) (c) {c};
    \node[highlight] at (-1, 0.2) (e) {e};

    \path (1) edge (3);
    \path (3) edge (4);
    \path (4) edge (5);
    \path (5) edge (1);
    
    \path (a) edge (c);    
    \path (c) edge (e);
    
    \path (1) edge (a);
    \path (3) edge (c);
    \path (4) edge (a);
    \path[highlight] (5) edge (e);

\end{tikzpicture}
    \end{minipage}
    \begin{minipage}{.07\linewidth}
    \centering
    $\Longrightarrow$
    \end{minipage}    
    \begin{minipage}{.39\linewidth}
    \centering
    \begin{tikzpicture}[node distance={30mm}, thick, main/.style = {draw, circle}, highlight/.style = {draw=blue, circle,text=blue}]

    \node[main] at (0, 2.3) (1) {1};
    \node[main] at (1.5, -2) (3) {3};
    \node[main] at (-1.5, -2) (4) {4};
    \node[main] at (-2.3, 0.4) (5) {5};

    \node[main] at (0, 1) (a) {a};
    \node[main] at (0.7, -1) (c) {c};

    \path (1) edge (3);
    \path (3) edge (4);
    \path (4) edge (5);
    \path (5) edge (1);
    
    \path (a) edge (c);    

    \path (1) edge (a);
    \path (3) edge (c);
    \path (4) edge (a);
    \path (5) edge (c);

\end{tikzpicture} 
    \end{minipage}
    \caption{Schritt 4: Kante zwischen Knoten e und Knoten 5 kontrahieren.}
    \label{fig:my_label}
\end{figure}


\begin{figure}[H]
    \centering
    \begin{minipage}{.39\linewidth}
    \centering
    \begin{tikzpicture}[node distance={30mm}, thick, main/.style = {draw, circle}, highlight/.style = {draw=blue, circle,text=blue}]

    \node[highlight] at (0, 2.3) (1) {1};
    \node[main] at (1.5, -2) (3) {3};
    \node[highlight] at (-1.5, -2) (4) {4};
    \node[main] at (-2.3, 0.4) (5) {5};

    \node[main] at (0, 1) (a) {a};
    \node[highlight] at (0.7, -1) (c) {c};

    \path (1) edge (3);
    \path (3) edge (4);
    \path (4) edge (5);
    \path (5) edge (1);
    
    \path (a) edge (c);    

    \path (1) edge (a);
    \path (3) edge (c);
    \path (4) edge (a);
    \path (5) edge (c);

\end{tikzpicture} 
    \end{minipage}
    \begin{minipage}{.07\linewidth}
    \centering
    $\Longrightarrow$
    \end{minipage}
    \begin{minipage}{.39\linewidth}
    \centering
    \begin{tikzpicture}[node distance={30mm}, thick, main/.style = {draw, circle}, highlight/.style = {draw=blue, circle,text=blue}]

    \node[main] at (0, 0) (1) {1};
    \node[main] at (2, 0) (4) {4};
    \node[main] at (4, 0) (c) {c};
    
    \node[main] at (0, -2) (5) {5};
    \node[main] at (2, -2) (3) {3};

    \node[main] at (4, -2) (a) {a};

    \path (1) edge (3);
    \path (3) edge (4);
    \path (4) edge (5);
    \path (5) edge (1);
    
    \path (a) edge (c);    

    \path (1) edge (a);
    \path (3) edge (c);
    \path (4) edge (a);
    \path (5) edge (c);

\end{tikzpicture} 
    \end{minipage}
    \caption{Schritt 5: Anordnung der Knoten zu der üblichen Darstellung des $K_{3,3}$ verschieben.}
    \label{fig:my_label}
\end{figure}

Da der Petersen-Graph, wie in dem vorherigen Beispiel gezeigt, den $K_{3,3}$ als Minor enthält ist er \textbf{nicht} planar.


\section{Anhang}

\subsection{Algorithmus von John Hopcroft und Robert Tarjan}
1974 haben John Hopcroft und Robert Tarjan einen Algorithmus zum Testen auf Planarität von Graphen vorgestellt.
Mithilfe von speziellen Datenstrukturen kann dieser Algorithmus in linearer Zeit ($O(n)$) einen Graphen auf Planarität testen.

\section{Quellen}

\begin{itemize}
    \item $[1]$  \textquote{Graphen- und Netzwerkoptimierung} Christina Buesing ISBN 978-3-8274-2422-8
    \item $[2]$  Wikipedia \begin{verbatim}
        https://de.wikipedia.org/wiki/Minor_(Graphentheorie)
        https://de.wikipedia.org/wiki/Kantenkontraktion
        https://de.wikipedia.org/wiki/Unterteilungsgraph
    \end{verbatim}
    \item $[3]$ \textquote{Graphen an allen Ecken und Kanten},
Lutz Volkmann,
Lehrstuhl 2 Mathematik RWTH-Aachen 2011,
Version 2
\begin{verbatim}
    https://www.math2.rwth-aachen.de/files/gt/buch/graphen_an_allen_ecken_und_kanten.pdf
\end{verbatim}

\end{itemize}

\end{document}
