\documentclass[aspectratio=169]{beamer}
\usepackage[ngerman]{babel}

\usepackage[utf8]{inputenc}
\usepackage{multicol}

\usetheme[sectionpage=none, subsectionpage=none,progressbar=frametitle]{metropolis}
\setbeamertemplate{frame numbering}[fraction]

\useoutertheme{metropolis}
\useinnertheme{metropolis}
\usefonttheme{metropolis}
\usecolortheme{seahorse}
\setbeamercolor{background canvas}{bg=white}

\usepackage{xcolor}

\definecolor{FHgreen}{HTML}{00978F}
\definecolor{FHgreenLight}{HTML}{6BC6C6}
\definecolor{FHgreenLightLight}{HTML}{AAEBEB}
\definecolor{FHgreenLightExtra}{HTML}{E8FFFF}
\definecolor{FHgreenLightMedium}{HTML}{b0e0e0}
\definecolor{FHorange}{HTML}{F49300}
\definecolor{FHorangeLight}{HTML}{FFEAC5}
\definecolor{FHorangeLightLight}{HTML}{FFD7A8}
\definecolor{FHgray}{HTML}{AAAAAA}
\definecolor{FHgrayLight}{HTML}{EBEBEB}
\definecolor{FHmint}{cmyk}{0.75,0,0.4,0}
\definecolor{FHmintDarkMedium}{cmyk}{0.75,0,0.4,0.10}
\definecolor{FHmintDark}{cmyk}{0.75,0,0.4,0.25}
\definecolor{FHmintDarkDark}{cmyk}{0.75,0,0.4,0.50}
\definecolor{FHmintDarkExtra}{cmyk}{0.75,0,0.4,0.75}
\definecolor{FHblack}{cmyk}{0,0,0,1.0}
\definecolor{FHblackLightMedium}{cmyk}{0,0,0,0.75}
\definecolor{FHblackLight}{cmyk}{0,0,0,0.50}
\definecolor{FHblackLightLight}{cmyk}{0,0,0,0.25}
\definecolor{FHblackLightExtra}{cmyk}{0,0,0,0.10}


\setbeamercolor{structure}{fg=black, bg=white}
\setbeamercolor{palette primary}{fg=black, bg=white}
\setbeamercolor{palette quaternary}{fg=red, bg=green}

\setbeamercolor{titlelike}{parent=palette primary}
\setbeamercolor*{author}{parent=titlelike}
\setbeamercolor*{date}{parent=titlelike}
\setbeamercolor*{institute}{parent=titlelike}
\setbeamercolor*{frametitle}{parent=palette primary}
\setbeamercolor*{frametitle}{bg=}
\setbeamercolor*{title in head/foot}{parent=titlelike}
\setbeamercolor*{title in head/foot}{bg=}
\setbeamercolor*{subsubsection title}{use=structure}
\setbeamercolor*{alerted text}{fg=FHmint, bg=white}
\setbeamercolor{local structure}{parent=structure}

\setbeamercolor{item}{parent=local structure}
\setbeamercolor{subitem}{parent=item}
\setbeamercolor{subsubitem}{parent=subitem}

\usepackage{csquotes}


\title{Minoren und der Satz von Kuratowski}
\author{Sven Kube}
\date{10. Juni 2022}

\setbeamercovered{transparent=5}

\begin{document}
\metroset{block=fill}
\setbeamertemplate{frame footer}{10. Juni 2022\text{ - Sven Kube - }\inserttitle}


\AtBeginSection[]
  {
    \ifnum \value{framenumber}>1
      \begin{frame}<beamer>
      \frametitle{Überblick}
      \tableofcontents[currentsection]
      \end{frame}
    \else
    \fi
  }


\begin{frame}
\titlepage
\begin{tikzpicture}[remember picture,overlay,shift={(current page.north east)}]
\node[anchor=north east,xshift=5pt,yshift=-0.3cm]{\includegraphics[width=1.4cm]{./FHAC.jpg}};
\end{tikzpicture}
\end{frame}

\begin{frame}[t]{Motivation} \vspace{4pt}

Bisher: Eulers Polyedersatz $\Rightarrow$ $K_{3,3}$ und $K_5$ nicht planar.

\vspace{1em}

Überlegung: 

\begin{itemize}
    \item Erweiterter $K_{3,3}$ oder $K_5$ ist auch nicht planar.
    \item $K_{3,3}$ und $K_5$ sind \enquote{Atome von Nichtplanarität} oder \enquote{kleinste Elemente nicht planarer Graphen}.
    \item Wenn in einem Graph der $K_{3,3}$ oder $K_5$ \textit{drin} ist $\Rightarrow$ Graph nicht planar.
\end{itemize}

\end{frame}

\begin{frame}[c]{Überblick}\vspace{0.6em}
\small{
\tableofcontents
}
\end{frame}

\section{Die Graphen $K_{3,3}$ und $K_5$ sind fast planar}
\subsection{}

\begin{frame}[c]{$K_{3,3}$ und $K_5$ sind fast planar} \vspace{4pt}

\begin{block}{Definition: Fast planar}
Ein Graph $G$ ist fast planar, wenn er durch löschen einer Kante planar wird.
\end{block}

\vspace{1em}
Gilt für $K_{3,3}$ und $K_5$, siehe folgende Beispiele.

\end{frame}

\begin{frame}[c]{$K_{3,3}$ ist fast planar} \vspace{4pt}
\vspace{-4em}
\begin{figure}[H]
    \centering
    \begin{minipage}{.3\linewidth}
    \centering
        \vspace{-2.2cm}
        \begin{tikzpicture}[scale=0.8,node distance={30mm}, thick, main/.style = {draw, circle}, highlight/.style = {draw=blue, circle,text=blue}]

    \node[main] at (0, 0) (1) {1};
    \node[main] at (2, 0) (2) {2};
    \node[main] at (4, 0) (3) {3};
    
    \node[main] at (0, -2) (4) {4};
    \node[main] at (2, -2) (5) {5};
    \node[main] at (4, -2) (6) {6};

    \path (1) edge (4);
    \path (1) edge (5);
    \path (1) edge (6);
    
    \path (2) edge (4);
    \path (2) edge (5);
    \path (2) edge (6);
    
    \path[highlight] (3) edge (4);
    \path (3) edge (5);
    \path (3) edge (6);
    
    \end{tikzpicture} 
    \end{minipage}
    \begin{minipage}{.07\linewidth}
    \centering
    \vspace{-2.2cm}
    $\Longrightarrow$
    \end{minipage}
    \begin{minipage}{.5\linewidth}
    \centering
    \begin{tikzpicture}[scale=0.8,node distance={30mm}, thick, main/.style = {draw, circle}, highlight/.style = {draw=blue, circle,text=blue}]

    \node[main] at (0, 0) (1) {1};
    \node[main] at (2, 0) (2) {2};
    \node[main] at (4, 0) (3) {3};
    
    \node[main] at (0, -2) (4) {4};
    \node[main] at (2, -2) (5) {5};
    \node[main] at (4, -2) (6) {6};

    \path (1) edge (4);
    \path (1) edge (5);
    \path (1) edge (6);
    
    \path (2) edge [out=130,in=180, distance=3cm ] (4);
    \path (2) edge [out=30,in=300, distance=7cm ] (5);
    \path (2) edge (6);
    
    \path (3) edge [out=330,in=310, distance=3cm ] (5);
    \path (3) edge (6);
    
    \end{tikzpicture} 
    \end{minipage}
    \vspace{-7em}
    \caption{Durch entfernen einer Kante wird der $K_{3,3}$ planar.}
    \label{fig:my_label}
\end{figure}

\end{frame}

\begin{frame}[c]{$K_5$ ist fast planar}
\begin{figure}[H]
    \centering
    \begin{minipage}{.4\linewidth}
        \centering
 \begin{tikzpicture}[scale=0.8,node distance={30mm}, thick, main/.style = {draw, circle}, highlight/.style = {draw=blue, circle,text=blue}]

    \node[main] at (0, 2.3) (1) {1};
    \node[main] at (2.3, 0.4) (2) {2};
    \node[main] at (1.5, -2) (3) {3};
    \node[main] at (-1.5, -2) (4) {4};
    \node[main] at (-2.3, 0.4) (5) {5};


    \path (1) edge (2);
    \path[blue] (1) edge (3);
    \path (1) edge (4);
    \path (2) edge (3);
    \path (2) edge (4);
    \path (2) edge (5);
    \path (3) edge (4);
    \path (3) edge (5);
    \path (4) edge (5);
    \path (5) edge (1);
    
\end{tikzpicture} 
\end{minipage}
\begin{minipage}{.07\linewidth}
\centering
$\Longrightarrow$
\end{minipage}
    \begin{minipage}{.4\linewidth}
        \centering
      
    \begin{tikzpicture}[scale=0.8,node distance={30mm}, thick, main/.style = {draw, circle}, highlight/.style = {draw=blue, circle,text=blue}]

    \node[main] at (0, 2.3) (1) {1};
    \node[main] at (2.3, 0.4) (2) {2};
    \node[main] at (1.5, -2) (3) {3};
    \node[main] at (-1.5, -2) (4) {4};
    \node[main] at (-2.3, 0.4) (5) {5};


    \path (1) edge (2);
    \path (1) edge [out=160,in=180, distance=3cm ] (4);
    \path (2) edge (3);
    \path (2) edge [out=300,in=320, distance=3cm ] (4);
    \path (2) edge (5);
    \path (3) edge (4);
    \path (3) edge (5);
    \path (4) edge (5);
    \path (5) edge (1);
    

\end{tikzpicture} 
\end{minipage}
\vspace{-1em}
    \caption{Durch entfernen einer Kante wird der $K_5$ planar.}
    \label{fig:my_label}
\end{figure}

\end{frame}




\section{Kantenkontraktion}

\subsection{}

\begin{frame}[t]{Kantenkontraktion}

\begin{figure}[H]
    \centering
    \begin{tabular}{@{}c@{}}
    \begin{tikzpicture}[scale = 0.7, node distance={30mm}, thick, main/.style = {draw, circle}, highlight/.style = {draw=blue, circle,text=blue}]
    
        \node[main] at (-2, 1)  (1)    {};
        \node[main] at (1.5, 0.8)  (2)    {};
        
        \node[highlight] at (0, 0)  (u)    {u};
        \node[highlight] at (0, -2)  (v)    {v};
        
        \node[main] at (-1, -3.5)  (3)    {};
        \node[main] at (0.3, -3.5)  (4)    {};
        \node[main] at (1.5, -2)  (5)    {};
    
        
        \path (1) edge (2);
        \path (1) edge (u);
        \path (2) edge (u);
        
        \path[blue] (u) edge node[right] {$e$} (v);
        
        \path (3) edge (v);
        \path (4) edge (v);
        \path (5) edge (v);
        
        \path (4) edge (5);
        
        \path[dashed] (1) edge  (-2.5, 1.5);
        \path[dashed] (1) edge  (-0.8, 1.6);
        
        \path[dashed] (2) edge  (1.5, 1.3);
        \path[dashed] (2) edge  (2, 0.8);
        
        
        \path[dashed] (3) edge  (-1.5, -3.5);
        \path[dashed] (3) edge  (-1, -4);
        \path[dashed] (4) edge  (0.3, -4);
       
        
        \path[dashed] (5) edge  (2, -2.5);
        \path[dashed] (5) edge  (2, -1.5);
        
        \draw [-stealth,draw=green](-0.2,-0.4) -- (-0.2,-0.9);
        \draw [-stealth,draw=green](-0.2,-1.6) -- (-0.2,-1.1);

    
    \end{tikzpicture} 
    \end{tabular}
    \hspace{0.5cm}
    \begin{tabular}{p{1cm}}
    \begin{center}
         $\Longrightarrow$
    \end{center}
    \end{tabular}
    \hspace{0.5cm}
    \begin{tabular}{@{}c@{}}
   \begin{tikzpicture}[scale = 0.7, node distance={30mm}, thick, main/.style = {draw, circle}, highlight/.style = {draw=blue, circle,text=blue}]
    
        \node[main] at (-2, 1)  (1)    {};
        \node[main] at (1.5, 0.8)  (2)    {};
        
        
        \node[highlight] at (0, -1)  (w)    {w};
        
        \node[main] at (-1, -3.5)  (3)    {};
        \node[main] at (0.3, -3.5)  (4)    {};
        \node[main] at (1.5, -2)  (5)    {};
    
        
        \path (1) edge (2);
        \path (1) edge (w);
        \path (2) edge (w);
        
        
        \path (3) edge (w);
        \path (4) edge (w);
        \path (5) edge (w);
        
        \path (4) edge (5);
        
        \path[dashed] (1) edge  (-2.5, 1.5);
        \path[dashed] (1) edge  (-0.8, 1.6);
        
        \path[dashed] (2) edge  (1.5, 1.3);
        \path[dashed] (2) edge  (2, 0.8);
        
        
        \path[dashed] (3) edge  (-1.5, -3.5);
        \path[dashed] (3) edge  (-1, -4);
        \path[dashed] (4) edge  (0.3, -4);
       
        
        \path[dashed] (5) edge  (2, -2.5);
        \path[dashed] (5) edge  (2, -1.5);
        
      

    
    \end{tikzpicture}
    \end{tabular}
    \label{fig:edge_contraction}
\end{figure}

\begin{block}{Definition: Kantenkontraktion}
\vspace{0.5em}
Aus einem Graph $G$ wird eine Kante $e$ entfernt und die beiden vorher durch die Kante $e$ verbundenen Knoten $u$ und $v$ zu einem neuen Knoten $w$ vereinigt.
\vspace{0.5em}
\end{block}

\end{frame}



\section{Unterteilungsgraph}
\subsection{}

\begin{frame}[c]{Unterteilungsgraph} \vspace{4pt}


\begin{figure}[H]
    \centering
    $G:$\hspace{0.5cm}
        \begin{tabular}{@{}c@{}}

    \begin{tikzpicture}[node distance={30mm}, thick, main/.style = {draw, circle}]
        \node[main] (1) {$u$}; 
        \node[main] (2) [below of=1] {$v$}; 

        \path (1) edge  node [right] {$e$} (2);
    \end{tikzpicture}
    \end{tabular}
    \hspace{1cm}$\Longrightarrow$\hspace{1cm}
    $G':$\hspace{0.5cm}
        \begin{tabular}{@{}c@{}}
    \begin{tikzpicture}[node distance={15mm}, thick, main/.style = {draw, circle}]
        \node[main] (1) {$u$}; 
        \node[draw, circle, blue] (2) [below of=1] {$w$}; 
        \node[main] (3) [below of=2] {$v$}; 

       \path[blue] (1) edge node [right] {$e_1$} (2);
       \path[blue] (2) edge node [right] {$e_2$} (3);
    
    \end{tikzpicture} 
    \end{tabular}
    \label{fig:edge_subdivision}
\end{figure}

\begin{block}{Definition: Unterteilungsgraph}
\vspace{0.5em}
Als Unterteilungsgraph eines Graphen $G$ wird ein Graph bezeichnet, der durch (null-, ein- oder mehrmalige) Kantenunterteilung in $G$ entsteht.
\vspace{0.5em}
\end{block}

\end{frame}


\section{Minor und topologischer Minor}

\subsection{Minor}



\begin{frame}[b]{Minor}

\onslide<1->{

\begin{figure}[H]
    \centering
    \begin{minipage}{.32\linewidth}
    \centering
\begin{tikzpicture}[node distance={30mm}, thick, main/.style = {draw, circle, fill}, highlight/.style = {draw=blue, circle,text=blue}]

    \node[main] at (0, 0) (1) {};
    \node[main] at (-1, -2) (2) {};
    \node[main] at (1, -2) (3) {};

    \path (1) edge (2);
    \path (2) edge (3);
    \path (3) edge (1);

\end{tikzpicture}
\vspace{0.6cm}
  \caption{$K_3$}
  

\end{minipage}
    \begin{minipage}{.32\linewidth}
    \centering
 \begin{tikzpicture}[node distance={30mm}, thick, main/.style = {draw, circle, fill}, small/.style = {draw, circle, fill, scale = 0.4}]

    \node[small] at (0, 0) (11) {};
    \node[small] at (0.3, -0.3) (12) {};
    \node[small] at (0.4, 0.2) (13) {};

    
    
    \node[small] at (-1, -2) (21) {};
    \node[small] at (-0.6, -1.9) (22) {};
    
    
    
    \node[small] at (1, -2) (31) {};
    \node[small] at (1.15, -1.3) (32) {};
    \node[small] at (1.5, -1.3) (33) {};
    \node[small] at (1.5, -1.8) (34) {};
    

    \path (11) edge (12);
    \path (11) edge (13);
    \path (11) edge (21);
    \path (32) edge (12);
    
    \path (32) edge (33);
    \path (32) edge (31);
    \path (34) edge (33);
    \path (34) edge (31);
    \path (31) edge (33);
    
    \path (22) edge (31);
    \path (22) edge (21);
    
    \draw (0.2,0) circle (15pt);
    
    \draw (-0.8, -1.95) ellipse (15pt and 10pt);
    
    \draw (1.25, -1.7) circle (20pt);

\end{tikzpicture}
  \caption{$G_1$}
  

\end{minipage}
\begin{minipage}{.32\linewidth}
\centering
 \begin{tikzpicture}[node distance={30mm}, thick, main/.style = {draw, circle, fill}, small/.style = {draw, circle, fill, scale = 0.4},gray/.style = {draw=gray, circle, fill=gray, scale = 0.4}]

    \node[small] at (0, 0) (11) {};
    \node[small] at (0.3, -0.3) (12) {};
    \node[small] at (0.4, 0.2) (13) {};
    
    \node[gray] at (1.2, -0.3) (X1) {};
    \path[gray] (13) edge (X1);
    \path[gray] (33) edge (X1);

    
    
    \node[small] at (-1, -2) (21) {};
    \node[small] at (-0.6, -1.9) (22) {};
    
    \node[gray] at (-0.8, -0.4) (X2) {};
    \node[gray] at (-1.7, -2) (X3) {};
    \node[gray] at (-1.3, -1.5) (X4) {};
    
    \path[gray] (11) edge (X2);
    \path[gray] (X2) edge (X4);
    \path[gray] (X3) edge (X4);
    
    \path[gray] (X3) edge (21);
    \path[gray] (X4) edge (21);

    
    
    \node[small] at (1, -2) (31) {};
    \node[small] at (1.15, -1.3) (32) {};
    \node[small] at (1.5, -1.3) (33) {};
    \node[small] at (1.5, -1.8) (34) {};
    

    \path (11) edge (12);
    \path (11) edge (13);
    \path (11) edge (21);
    \path (32) edge (12);
    
    \path (32) edge (33);
    \path (32) edge (31);
    \path (34) edge (33);
    \path (34) edge (31);
    \path (31) edge (33);
    
    \path (22) edge (31);
    \path (22) edge (21);
    
    \draw (0.2,0) circle (15pt);
    
    \draw (-0.8, -1.95) ellipse (15pt and 10pt);
    
    \draw (1.25, -1.7) circle (20pt);

\end{tikzpicture} 
  \caption{$G_2$}
  

\end{minipage}
\end{figure}

}

\setbeamercovered{transparent=20}

\only<2->{
\begin{block}{Definition: Minor}
\vspace{0.5em}
Ein Graph $G$ heißt \textit{Minor} von $H$, wenn $H$ einen Teilgraph enthält, aus dem durch Kantenkontraktion $G$ hervorgeht.
\vspace{0.5em}
\end{block}
}
\end{frame}







\subsection{Topologischer Minor}

\begin{frame}[b]{Topologischer Minor}

\onslide<1->{
\begin{figure}[H]
    \centering
    \begin{minipage}{.32\linewidth}
    \centering
\begin{tikzpicture}[node distance={30mm}, thick, main/.style = {draw, circle, fill}, highlight/.style = {draw=blue, circle,text=blue}]

    \node[main] at (0, 0) (1) {};
    \node[main] at (-1, -2) (2) {};
    \node[main] at (1, -2) (3) {};

    \path (1) edge (2);
    \path (2) edge (3);
    \path (3) edge (1);

\end{tikzpicture}
  \caption{$K_3$}
  

\end{minipage}
    \begin{minipage}{.32\linewidth}
    \centering
   \begin{tikzpicture}[node distance={30mm}, thick, main/.style = {draw, circle, fill}, small/.style = {draw, circle, fill, scale = 0.4}]

    \node[main] at (0, 0) (1) {};
    
    \node[small] at (-0.5, -1) (a) {};
    
    \node[main] at (-1, -2) (2) {};
    
    \node[small] at (0.65, -1.3) (a) {};
    
    \node[main] at (1, -2) (3) {};
    
    \node[small] at (-0.5, -2) (c) {};
    \node[small] at (0.3, -2) (d) {};

    \path (1) edge (2);
    \path (2) edge (3);
    \path (3) edge (1);

\end{tikzpicture}
  \caption{$G_1$}
 

\end{minipage}
\begin{minipage}{.32\linewidth}
\centering
\begin{tikzpicture}[node distance={30mm}, thick, main/.style = {draw, circle, fill}, small/.style = {draw, circle, fill, scale = 0.4}, gray/.style = {draw=gray, circle, fill=gray,}]

    \node[main] at (0, 0) (1) {};
    
    
    \node[small] at (-0.5, -1) (a) {};
    
    \node[main] at (-1, -2) (2) {};
    
    \node[small] at (0.65, -1.3) (a) {};
    
    \node[main] at (1, -2) (3) {};
    
    \node[small] at (-0.5, -2) (c) {};
    \node[small] at (0.3, -2) (d) {};
    
    
    
    \node[gray] at (-1.5, 0) (z) {};
    \node[gray] at (2, 0) (y) {};
    \node[gray] at (2.5, -2) (x) {};

    

    \path (1) edge (2);
    \path (2) edge (3);
    \path (3) edge (1);
    
    \path[gray] (3) edge (x);
    \path[gray] (1) edge (y);
    \path[gray] (3) edge (y);
    \path[gray] (1) edge (z);

\end{tikzpicture} 
  \caption{$G_2$}

\end{minipage}
\end{figure}
}


\only<2->{
\begin{block}{Definition: Topologischer Minor}
\vspace{0.5em}
Ein Graph $G$ heißt \textit{topologischer Minor} von $H$, wenn $H$ einen Unterteilungsgraphen von $G$ enthält.
\vspace{0.5em}
\end{block}
}


\end{frame}


\subsection{Minorenrelation}

\begin{frame}[t]{Minorenrelation} \vspace{4pt}

\begin{block}{Definition: Minorenrelation}
\vspace{0.5em}
$ G \preccurlyeq H :\Longleftrightarrow G$ ist Minor von $H$.
\vspace{0.5em}
\end{block}
\vspace{0.5em}
\begin{enumerate}
    \item reflexiv, transitiv und antisymetrisch.\\
    Definiert Ordnungsrelation auf den endlichen Graphen.
    \item jeder topologische Minor eines Graphen $G$ ist auch ein Minor von $G$.
    \item \textbf{nicht} jeder Minor eines Graphen $G$ auch ein topologischer Minor von $G$.
\end{enumerate}

\end{frame}

\begin{frame}[b]{Beispiel: Nicht jeder Minor ist auch topologischer Minor} \vspace{4pt}

\begin{figure}[H]
    \centering
    \begin{minipage}{.32\linewidth}
    \centering

\onslide<2>{
\begin{tikzpicture}[node distance={30mm}, thick, main/.style = {draw, circle, fill}, highlight/.style = {draw=blue, circle,text=blue}]

    \node[main] at (0, 0) (1) {};
    \node[main] at (-3, 0) (2) {};
    \node[main] at (0, -3) (3) {};
    \node[main] at (-3, -3) (4) {};
    
    \node[main] at (-1.5, -1.5) (z) {};
    
    \node[draw=blue, circle,fill=blue] at (-0.75, -0.75) (a) {};
    \node[draw=blue, circle,fill=blue] at (-2.25, -0.75) (b) {};
    \node[draw=blue, circle,fill=blue] at (-0.75, -2.25) (c) {};
    \node[draw=blue, circle,fill=blue] at (-2.25, -2.25) (d) {};

    \path (1) edge (a);
    \path (2) edge (b);
    \path (3) edge (c);
    \path (4) edge (d);
    
    \path (z) edge (a);
    \path (z) edge (b);
    \path (z) edge (c);
    \path (z) edge (d);

    
    \path (1) edge (2);
    \path (1) edge (3);
    
    \path (4) edge (2);
    \path (4) edge (3);

\end{tikzpicture}
  \caption{$F$}
  }


\end{minipage}
    \begin{minipage}{.32\linewidth}
    \centering

\begin{tikzpicture}[node distance={30mm}, thick, main/.style = {draw, circle, fill}, highlight/.style = {draw=blue, circle,text=blue}]

    \node[main] at (0, 0) (1) {};
    \node[main] at (-3, 0) (2) {};
    \node[main] at (0, -3) (3) {};
    \node[main] at (-3, -3) (4) {};
    
    \node[draw=blue, circle,fill=blue] at (-1.5, -1.5) (5) {};

    \path (1) edge (5);
    \path (2) edge (5);
    \path (3) edge (5);
    \path (4) edge (5);
    
    \path (1) edge (2);
    \path (1) edge (3);
    
    \path (4) edge (2);
    \path (4) edge (3);

\end{tikzpicture}
  \caption{$G$}


\end{minipage}
    \begin{minipage}{.32\linewidth}
    \centering
\onslide<3->{

 \begin{tikzpicture}[node distance={30mm}, thick, main/.style = {draw, circle, fill}, highlight/.style = {draw=blue, circle,text=blue}]

    \node[main] at (0, 0) (1) {};
    \node[main] at (-3, 0) (2) {};
    \node[main] at (0, -3) (3) {};
    \node[main] at (-3, -3) (4) {};
    
    \node[draw=blue, circle,fill=blue] at (-1, -1) (a) {};
    \node[draw=blue, circle,fill=blue] at (-2, -1) (b) {};
    \node[draw=blue, circle,fill=blue] at (-1, -2) (c) {};
    \node[draw=blue, circle,fill=blue] at (-2, -2) (d) {};

    \path (a) edge (b);
    \path (a) edge (c);
    \path (d) edge (b);
    \path (d) edge (c);
    
    \path (1) edge (a);
    \path (2) edge (b);
    \path (3) edge (c);
    \path (4) edge (d);
    
    \path (1) edge (2);
    \path (1) edge (3);
    
    \path (4) edge (2);
    \path (4) edge (3);

\end{tikzpicture}
  \caption{$H$}
  }


\end{minipage}
\end{figure}


\end{frame}

\section{Satz von Kuratowski}

\subsection{}


\begin{frame}[c]{Satz von Kuratowski} \vspace{4pt}

Charakterisierung aller planaren Graphen mit Hilfe des $K_{3,3}$, $K_5$ und Minoren.

\vspace{1em}

\onslide<2->{
\begin{block}{Satz von Kuratowski \scriptsize{(Kazimierz Kuratowski, 1930)}}
\vspace{0.5em}
\textit{Ein Graph ist genau dann planar, wenn er weder einen $K_{3,3}$ noch einen $K_5$ als Minor enthält.}
\vspace{0.5em}
\end{block}
}

\onslide<3->{
\small
\textit{Bemerkung:} Ebenfalls gilt, ein Graph $G$ ist genau dann planar ist, wenn er weder den $K_{3,3}$ noch den $K_5$ als topologischen Minor enthält.\\
}

\end{frame}

\begin{frame}[t]{Satz von Kuratowski} \vspace{4pt}

Aussagen des Satzes von Kuratowski:


\begin{enumerate}
    \onslide<2->{
    \item{
    Wenn ein Graph $G$ den $K_{3,3}$ oder $K_5$ als Minor enthält, ist der Graph $G$ \textbf{nicht} planar.
    
    \vspace{0.5em}
    $\Rightarrow$ Intuitiv klar.
    }\\
    }
    \vspace{0.3em}
    \onslide<3->{
    \item{
    Wenn eine Graph $G$ nicht planar ist, enthält er den $K_{3,3}$ oder $K_5$ als Minor.
    
    \vspace{0.5em}
    $\Rightarrow$ \textbf{Nicht} intuitiv klar.
    }
    }
\end{enumerate}


\end{frame}

\subsection{Anwendung beim Petersen-Graph}

\begin{frame}[t]{Anwendung beim Petersen-Graph} \vspace{4pt}

\begin{columns}
\column{0.5\textwidth}


\onslide<1->{
Ist der $K_5$ Minor des Petersen-Graph?\\
}
\onslide<2->{
\vspace{0.3em}
$\Rightarrow$ \textbf{Ja}\\
}
\vspace{0.7em}
\onslide<3->{
Ist der $K_{3,3}$ Minor des Petersen-Graph?\\
}



\column{0.5\textwidth}

\only<1>{
\begin{figure}[H]
    \centering
    \begin{tikzpicture}[node distance={30mm}, thick, main/.style = {draw, circle}, highlight/.style = {draw=blue, circle,text=blue}]

    \node[main] at (0, 2.3) (1) {1};
    \node[main] at (2.3, 0.4) (2) {2};
    \node[main] at (1.5, -2) (3) {3};
    \node[main] at (-1.5, -2) (4) {4};
    \node[main] at (-2.3, 0.4) (5) {5};

    \node[main] at (0, 1) (a) {a};
    \node[main] at (1, 0.2) (b) {b};
    \node[main] at (0.7, -1) (c) {c};
    \node[main] at (-0.7, -1) (d) {d};
    \node[main] at (-1, 0.2) (e) {e};

    \path (1) edge (2);
    \path (2) edge (3);
    \path (3) edge (4);
    \path (4) edge (5);
    \path (5) edge (1);
    
    \path (a) edge (c);
    \path (a) edge (d);
    \path (b) edge (e);
    \path (b) edge (d);
    \path (c) edge (e);
    
    \path (1) edge (a);
    \path (2) edge (b);
    \path (3) edge (c);
    \path (4) edge (d);
    \path (5) edge (e);

\end{tikzpicture} 
 \vspace{1em}
    \caption{Petersen-Graph}
    \label{fig:my_label}
\end{figure}
}

\only<2>{
\begin{figure}[H]
    \centering
    \begin{tikzpicture}[node distance={30mm}, thick, main/.style = {draw, circle}, highlight/.style = {draw=blue, circle,text=blue}]

    \node[main] at (0, 2.3) (1) {1};
    \node[main] at (2.3, 0.4) (2) {2};
    \node[main] at (1.5, -2) (3) {3};
    \node[main] at (-1.5, -2) (4) {4};
    \node[main] at (-2.3, 0.4) (5) {5};

    \node[main] at (0, 1) (a) {a};
    \node[main] at (1, 0.2) (b) {b};
    \node[main] at (0.7, -1) (c) {c};
    \node[main] at (-0.7, -1) (d) {d};
    \node[main] at (-1, 0.2) (e) {e};

    \path (1) edge (2);
    \path (2) edge (3);
    \path (3) edge (4);
    \path (4) edge (5);
    \path (5) edge (1);
    
    \path (a) edge (c);
    \path (a) edge (d);
    \path (b) edge (e);
    \path (b) edge (d);
    \path (c) edge (e);
    
    \path[highlight] (1) edge (a);
    \path[highlight] (2) edge (b);
    \path[highlight] (3) edge (c);
    \path[highlight] (4) edge (d);
    \path[highlight] (5) edge (e);

\end{tikzpicture} 
 \vspace{1em}
    \caption{Petersen-Graph}
    \label{fig:my_label}
\end{figure}
}

\only<3>{
\begin{figure}[H]
    \centering
    \begin{tikzpicture}[node distance={30mm}, thick, main/.style = {draw, circle}, highlight/.style = {draw=blue, circle,text=blue}]

    \node[main] at (0, 2.3) (1) {1};
    \node[main] at (2.3, 0.4) (2) {2};
    \node[main] at (1.5, -2) (3) {3};
    \node[main] at (-1.5, -2) (4) {4};
    \node[main] at (-2.3, 0.4) (5) {5};

    \node[main] at (0, 1) (a) {a};
    \node[main] at (1, 0.2) (b) {b};
    \node[main] at (0.7, -1) (c) {c};
    \node[main] at (-0.7, -1) (d) {d};
    \node[main] at (-1, 0.2) (e) {e};

    \path (1) edge (2);
    \path (2) edge (3);
    \path (3) edge (4);
    \path (4) edge (5);
    \path (5) edge (1);
    
    \path (a) edge (c);
    \path (a) edge (d);
    \path (b) edge (e);
    \path (b) edge (d);
    \path (c) edge (e);
    
    \path (1) edge (a);
    \path (2) edge (b);
    \path (3) edge (c);
    \path (4) edge (d);
    \path (5) edge (e);

\end{tikzpicture} 
 \vspace{1em}
    \caption{Petersen-Graph}
    \label{fig:my_label}
\end{figure}
}

\end{columns}


\end{frame}


\begin{frame}[t]{Anwendung beim Petersen-Graph} \vspace{4pt}

\only<1>{
\begin{figure}[H]
    \centering
    \begin{minipage}{.39\linewidth}
    \centering
\begin{tikzpicture}[node distance={30mm}, thick, main/.style = {draw, circle}, highlight/.style = {draw=blue, circle,text=blue}]

    \node[main] at (0, 2.3) (1) {1};
    \node[main] at (2.3, 0.4) (2) {2};
    \node[main] at (1.5, -2) (3) {3};
    \node[main] at (-1.5, -2) (4) {4};
    \node[main] at (-2.3, 0.4) (5) {5};

    \node[main] at (0, 1) (a) {a};
    \node[highlight] at (1, 0.2) (b) {b};
    \node[main] at (0.7, -1) (c) {c};
    \node[main] at (-0.7, -1) (d) {d};
    \node[main] at (-1, 0.2) (e) {e};

    \path (1) edge (2);
    \path (2) edge (3);
    \path (3) edge (4);
    \path (4) edge (5);
    \path (5) edge (1);
    
    \path (a) edge (c);
    \path (a) edge (d);
    \path[highlight] (b) edge (e);
    \path[highlight] (b) edge (d);
    \path (c) edge (e);
    
    \path (1) edge (a);
    \path[highlight] (2) edge (b);
    \path (3) edge (c);
    \path (4) edge (d);
    \path (5) edge (e);

\end{tikzpicture} 
\end{minipage}
\begin{minipage}{.07\linewidth}
\centering
$\Longrightarrow$
\end{minipage}
\begin{minipage}{.39\linewidth}
\centering
\begin{tikzpicture}[node distance={30mm}, thick, main/.style = {draw, circle}, highlight/.style = {draw=blue, circle,text=blue}]

    \node[main] at (0, 2.3) (1) {1};
    \node[main] at (2.3, 0.4) (2) {2};
    \node[main] at (1.5, -2) (3) {3};
    \node[main] at (-1.5, -2) (4) {4};
    \node[main] at (-2.3, 0.4) (5) {5};

    \node[main] at (0, 1) (a) {a};
    \node[main] at (0.7, -1) (c) {c};
    \node[main] at (-0.7, -1) (d) {d};
    \node[main] at (-1, 0.2) (e) {e};

    \path (1) edge (2);
    \path (2) edge (3);
    \path (3) edge (4);
    \path (4) edge (5);
    \path (5) edge (1);
    
    \path (a) edge (c);
    \path (a) edge (d);
    \path (c) edge (e);
    
    \path (1) edge (a);
    \path (3) edge (c);
    \path (4) edge (d);
    \path (5) edge (e);

\end{tikzpicture} 
\end{minipage}
 \vspace{1em}
    \caption{Schritt 1: Knoten B wird gelöscht.}
    \label{fig:my_label}
\end{figure}
}

\only<2>{
\begin{figure}[H]
    \centering
    \begin{minipage}{.39\linewidth}
    \centering
    \begin{tikzpicture}[node distance={30mm}, thick, main/.style = {draw, circle}, highlight/.style = {draw=blue, circle,text=blue}]

    \node[main] at (0, 2.3) (1) {1};
    \node[main] at (2.3, 0.4) (2) {2};
    \node[main] at (1.5, -2) (3) {3};
    \node[main] at (-1.5, -2) (4) {4};
    \node[main] at (-2.3, 0.4) (5) {5};

    \node[main] at (0, 1) (a) {a};
    \node[main] at (0.7, -1) (c) {c};
    \node[highlight] at (-0.7, -1) (d) {d};
    \node[main] at (-1, 0.2) (e) {e};

    \path (1) edge (2);
    \path (2) edge (3);
    \path (3) edge (4);
    \path (4) edge (5);
    \path (5) edge (1);
    
    \path (a) edge (c);
    \path[highlight] (a) edge (d);
    \path (c) edge (e);
    
    \path (1) edge (a);
    \path (3) edge (c);
    \path (4) edge (d);
    \path (5) edge (e);

\end{tikzpicture} 
    \end{minipage}
    \begin{minipage}{.07\linewidth}
    \centering
    $\Longrightarrow$
    \end{minipage}
    \begin{minipage}{.39\linewidth}
    \centering
    \begin{tikzpicture}[node distance={30mm}, thick, main/.style = {draw, circle}, highlight/.style = {draw=blue, circle,text=blue}]

    \node[main] at (0, 2.3) (1) {1};
    \node[main] at (2.3, 0.4) (2) {2};
    \node[main] at (1.5, -2) (3) {3};
    \node[main] at (-1.5, -2) (4) {4};
    \node[main] at (-2.3, 0.4) (5) {5};

    \node[main] at (0, 1) (a) {a};
    \node[main] at (0.7, -1) (c) {c};
    \node[main] at (-1, 0.2) (e) {e};

    \path (1) edge (2);
    \path (2) edge (3);
    \path (3) edge (4);
    \path (4) edge (5);
    \path (5) edge (1);
    
    \path (a) edge (c);    \path (c) edge (e);
    
    \path (1) edge (a);
    \path (3) edge (c);
    \path (4) edge (a);
    \path (5) edge (e);

\end{tikzpicture} 
    \end{minipage}
     \vspace{1em}
    \caption{Schritt 2: Kante zwischen Knoten d und Knoten a kontrahieren.}
    \label{fig:my_label}
\end{figure}

}
\only<3>{
\begin{figure}[H]
    \centering
    \begin{minipage}{.39\linewidth}
    \centering
    \begin{tikzpicture}[node distance={30mm}, thick, main/.style = {draw, circle}, highlight/.style = {draw=blue, circle,text=blue}]

    \node[main] at (0, 2.3) (1) {1};
    \node[highlight] at (2.3, 0.4) (2) {2};
    \node[main] at (1.5, -2) (3) {3};
    \node[main] at (-1.5, -2) (4) {4};
    \node[main] at (-2.3, 0.4) (5) {5};

    \node[main] at (0, 1) (a) {a};
    \node[main] at (0.7, -1) (c) {c};
    \node[main] at (-1, 0.2) (e) {e};

    \path (1) edge (2);
    \path[highlight] (2) edge (3);
    \path (3) edge (4);
    \path (4) edge (5);
    \path (5) edge (1);
    
    \path (a) edge (c);    \path (c) edge (e);
    
    \path (1) edge (a);
    \path (3) edge (c);
    \path (4) edge (a);
    \path (5) edge (e);

\end{tikzpicture}
    \end{minipage}
    \begin{minipage}{.07\linewidth}
    \centering
    $\Longrightarrow$
    \end{minipage}
    \begin{minipage}{.39\linewidth}
    \centering
    \begin{tikzpicture}[node distance={30mm}, thick, main/.style = {draw, circle}, highlight/.style = {draw=blue, circle,text=blue}]

    \node[main] at (0, 2.3) (1) {1};
    \node[main] at (1.5, -2) (3) {3};
    \node[main] at (-1.5, -2) (4) {4};
    \node[main] at (-2.3, 0.4) (5) {5};

    \node[main] at (0, 1) (a) {a};
    \node[main] at (0.7, -1) (c) {c};
    \node[main] at (-1, 0.2) (e) {e};

    \path (1) edge (3);
    \path (3) edge (4);
    \path (4) edge (5);
    \path (5) edge (1);
    
    \path (a) edge (c);    
    \path (c) edge (e);
    
    \path (1) edge (a);
    \path (3) edge (c);
    \path (4) edge (a);
    \path (5) edge (e);

\end{tikzpicture} 
    \end{minipage}
 \vspace{1em}
    \caption{Schritt 3: Kante zwischen Knoten 2 und Knoten 3 kontrahieren.}
    \label{fig:my_label}
\end{figure}

}
\only<4>{
\begin{figure}[H]
    \centering
    \begin{minipage}{.39\linewidth}
    \centering
    \begin{tikzpicture}[node distance={30mm}, thick, main/.style = {draw, circle}, highlight/.style = {draw=blue, circle,text=blue}]

    \node[main] at (0, 2.3) (1) {1};
    \node[main] at (1.5, -2) (3) {3};
    \node[main] at (-1.5, -2) (4) {4};
    \node[main] at (-2.3, 0.4) (5) {5};

    \node[main] at (0, 1) (a) {a};
    \node[main] at (0.7, -1) (c) {c};
    \node[highlight] at (-1, 0.2) (e) {e};

    \path (1) edge (3);
    \path (3) edge (4);
    \path (4) edge (5);
    \path (5) edge (1);
    
    \path (a) edge (c);    
    \path (c) edge (e);
    
    \path (1) edge (a);
    \path (3) edge (c);
    \path (4) edge (a);
    \path[highlight] (5) edge (e);

\end{tikzpicture}
    \end{minipage}
    \begin{minipage}{.07\linewidth}
    \centering
    $\Longrightarrow$
    \end{minipage}    
    \begin{minipage}{.39\linewidth}
    \centering
    \begin{tikzpicture}[node distance={30mm}, thick, main/.style = {draw, circle}, highlight/.style = {draw=blue, circle,text=blue}]

    \node[main] at (0, 2.3) (1) {1};
    \node[main] at (1.5, -2) (3) {3};
    \node[main] at (-1.5, -2) (4) {4};
    \node[main] at (-2.3, 0.4) (5) {5};

    \node[main] at (0, 1) (a) {a};
    \node[main] at (0.7, -1) (c) {c};

    \path (1) edge (3);
    \path (3) edge (4);
    \path (4) edge (5);
    \path (5) edge (1);
    
    \path (a) edge (c);    

    \path (1) edge (a);
    \path (3) edge (c);
    \path (4) edge (a);
    \path (5) edge (c);

\end{tikzpicture} 
    \end{minipage}
     \vspace{1em}
    \caption{Schritt 4: Kante zwischen Knoten e und Knoten 5 kontrahieren.}
    \label{fig:my_label}
\end{figure}

}
\only<5>{
\begin{figure}[H]
    \centering
    \begin{minipage}{.39\linewidth}
    \centering
    \begin{tikzpicture}[node distance={30mm}, thick, main/.style = {draw, circle}, highlight/.style = {draw=blue, circle,text=blue}]

    \node[highlight] at (0, 2.3) (1) {1};
    \node[main] at (1.5, -2) (3) {3};
    \node[highlight] at (-1.5, -2) (4) {4};
    \node[main] at (-2.3, 0.4) (5) {5};

    \node[main] at (0, 1) (a) {a};
    \node[highlight] at (0.7, -1) (c) {c};

    \path (1) edge (3);
    \path (3) edge (4);
    \path (4) edge (5);
    \path (5) edge (1);
    
    \path (a) edge (c);    

    \path (1) edge (a);
    \path (3) edge (c);
    \path (4) edge (a);
    \path (5) edge (c);

\end{tikzpicture} 
    \end{minipage}
    \begin{minipage}{.07\linewidth}
    \centering
    $\Longrightarrow$
    \end{minipage}
    \begin{minipage}{.39\linewidth}
    \centering
    \begin{tikzpicture}[node distance={30mm}, thick, main/.style = {draw, circle}, highlight/.style = {draw=blue, circle,text=blue}]

    \node[main] at (0, 0) (1) {1};
    \node[main] at (2, 0) (4) {4};
    \node[main] at (4, 0) (c) {c};
    
    \node[main] at (0, -2) (5) {5};
    \node[main] at (2, -2) (3) {3};

    \node[main] at (4, -2) (a) {a};

    \path (1) edge (3);
    \path (3) edge (4);
    \path (4) edge (5);
    \path (5) edge (1);
    
    \path (a) edge (c);    

    \path (1) edge (a);
    \path (3) edge (c);
    \path (4) edge (a);
    \path (5) edge (c);

\end{tikzpicture} 
    \end{minipage}
     \vspace{1em}
    \caption{Schritt 5: Anordnung der Knoten anpassen.}
    \label{fig:my_label}
\end{figure}
}

\only<6>{
\vspace{-0.8em}
\begin{figure}[H]
    \centering
    \begin{minipage}{.39\linewidth}
    \centering
    \begin{tikzpicture}[scale=0.90, node distance={30mm}, thick, main/.style = {draw, circle}, highlight/.style = {draw=blue, circle,text=blue}]

    \node[highlight] at (0, 2.3) (1) {1};
    \node[main] at (1.5, -2) (3) {3};
    \node[highlight] at (-1.5, -2) (4) {4};
    \node[main] at (-2.3, 0.4) (5) {5};

    \node[main] at (0, 1) (a) {a};
    \node[highlight] at (0.7, -1) (c) {c};

    \path (1) edge (3);
    \path (3) edge (4);
    \path (4) edge (5);
    \path (5) edge (1);
    
    \path (a) edge (c);    

    \path (1) edge (a);
    \path (3) edge (c);
    \path (4) edge (a);
    \path (5) edge (c);

\end{tikzpicture} 
    \end{minipage}
    \begin{minipage}{.07\linewidth}
    \centering
    $\Longrightarrow$
    \end{minipage}
    \begin{minipage}{.39\linewidth}
    \centering
    \begin{tikzpicture}[scale=0.90, node distance={30mm}, thick, main/.style = {draw, circle}, highlight/.style = {draw=blue, circle,text=blue}]

    \node[main] at (0, 0) (1) {1};
    \node[main] at (2, 0) (4) {4};
    \node[main] at (4, 0) (c) {c};
    
    \node[main] at (0, -2) (5) {5};
    \node[main] at (2, -2) (3) {3};

    \node[main] at (4, -2) (a) {a};

    \path (1) edge (3);
    \path (3) edge (4);
    \path (4) edge (5);
    \path (5) edge (1);
    
    \path (a) edge (c);    

    \path (1) edge (a);
    \path (3) edge (c);
    \path (4) edge (a);
    \path (5) edge (c);

\end{tikzpicture} 
    \end{minipage}
    \label{fig:my_label}
\end{figure}

Hat auch den $K_{3,3}$ als Minor.\\
\vspace{0.1em}
$\Rightarrow$ \textbf{nicht planar.}\\

}

\end{frame}

\section*{Zusammenfassung}

\subsection{}

\begin{frame}[c]{Zusammenfassung}

\begin{itemize}
    \item Fast Planarität von Graphen am Beispiel des $K_{3,3}$ und $K_5$
    \item Kantenkontraktion
    \item Unterteilungsgraph
    \item Minor
    \item Topologischer Minor
    \item Minorenrelation
    \item Satz von Kuratowski
    \item Anwendung des Satzes von Kuratowski am Petersen Graph
\end{itemize}

\end{frame}

\end{document}
